\documentclass[12pt]{article}
\usepackage[utf8]{inputenc}
\usepackage{graphicx}
\usepackage{geometry}
\usepackage{ragged2e}
\geometry{a4paper, left=30mm, top=30mm, right=20mm, bottom=20mm}
\title{Padawan 2021 - Módulo C\#}
\date{2021\\ Maio}
\author{Bruno Anhaia}
\pagenumbering{arabic}

\begin{document}

\maketitle

\section{Calculadora}
\justifying
Hirosberto, que está participando do Padawan, só sabe pensar em problemas matemáticos, por isso, para fixar seus conhecimentos na linguagem C\# e orientação a objetos decidiu criar uma calculadora com os seguintes requisitos:

\begin{itemize}
    \item O software deverá aceitar como entrada uma \textbf{única} string contendo dois números e um caractere utilizado como operador, separados por um único espaço entre eles, ou seja, no seguinte formato:

          \begin{center}
              numero \space\space operador \space\space numero;
          \end{center}

          Exemplo: 1 + 2

    \item Deverá suportar as seguintes operações:

          \begin{center}\begin{tabular}{c | c}
                  \hline
                  \textbf{Operação} & \textbf{Símbolo} \\
                  \hline
                  Adição            & +                \\
                  Subtração         & -                \\
                  Multiplicação     & *                \\
                  Divisão           & /                \\
                  \hline
              \end{tabular}\end{center}

    \item Ao receber uma entrada inválida o program deverá exibir a seguinte mensagem ``Entrada inválida, por favor insira uma entrada válida''.
    \item O programa deverá ser executado até que seja inserida a string ``SAIR''.
    \item As exceções deverão ser tratadas.
    \item Os resultados deverão ser exibidos com 3 casas decimais de precisão.
\end{itemize}

Faça como Hirosberto, crie um programa com as mesmas funcionalidades.
Veja abaixo alguns exemplos de entradas e saídas esperadas:

\begin{center}
    \begin{tabular}{c | c}
        \hline
        \textbf{Entrada} & \textbf{Saída}                                 \\
        \hline
        7.99 + 8.21      & 16.200                                         \\
        1 - 100.4        & -99.400                                        \\
        45.091 * 20.754  & 935.819                                        \\
        1 / 3            & 0.333                                          \\
        30.0 / 0         & ``Não é possível realizar a divisão por zero'' \\
        Entrada vazia    & ``Entrada inválida, tente novamente''          \\
        1.00+100         & ``Entrada inválida, tente novamente''          \\

        \hline
    \end{tabular}
\end{center}

\end{document}
